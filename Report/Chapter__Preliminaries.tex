
\chapter{Preliminaries}



\section{Modeling}

\subsection{Flux dynamics}
We utilize the discrete time dynamics of the stator and rotor fluxes $\b{\psi}_s$ and $\b{\psi}_r$ from \cite{gey_book}:
\begin{align*}
	\b{\psi}_s(k+1) &= \b{A}_1\b{\psi}_s(k) + \b{B}_1\b{\psi}_r(k) + \b{B}_2\b{u}(k), \\
	\b{\psi}_r(k+1) &= \b{B}_3\b{\psi}_s(k) + \b{A}_2\b{\psi}_r(k),
\end{align*}
where $\b{u}(k) \in \{-1,0,1\}^3$ is the normalized input.
These dynamics can be summarized to
\begin{equation*}
	\b{x}(k+1) = \b{Ax}(k) + \b{Bu}(k),
\end{equation*}
where
\begin{equation*}
	A = \left[\begin{array}{cc}
		\b{A}_1 & \b{B}_1 \\
		\b{B}_3 & \b{A}_2
	\end{array}\right],\ \b{B} = \left[\begin{array}{c}
		\b{B}_2 \\
		\b{0}
	\end{array}\right].
\end{equation*}
The values of the rotor and stator flux at time step $k+l$, given the state $x(k)$, all inputs $\b{u}(k),\ \b{u}(k+1),\ldots,\ \b{u}(k+l-1)$, and $l \geq 2$ can now be computed as
\begin{align}\label{eq:0}
	\b{\psi}_s(k+l) &= \left[\b{A}_1\ \b{B}_1\right]\b{A}^{l-1}\b{x}(k) + \left[\b{A}_1\ \b{B}_1\right]\b{A}^{l-2}\b{Bu}(k) + \\
	&\ + \left[\b{A}_1\ \b{B}_1\right]\b{A}^{l-3}\b{Bu}(k+1) + \ldots + \left[\b{A}_1\ \b{B}_1\right]\b{Bu}(k+l-1) + \b{B}_2\b{u}(k+l) \nonumber
\end{align}
and
\begin{align} \label{eq:1}
	\b{\psi}_r(k+l) &= \left[\b{B}_3\ \b{A}_2\right]\b{A}^{l-1}\b{x}(k) + \left[\b{B}_3\ \b{A}_2\right]\b{A}^{l-2}\b{Bu}(k) + \\
	&\ + \left[\b{B}_3\ \b{A}_2\right]\b{A}^{l-3}\b{Bu}(k+1) + \ldots + \left[\b{B}_3\ \b{A}_2\right]\b{Bu}(k+l-1). \nonumber
\end{align}

\subsection{Torque and absolute stator flux}

The absolute stator flux is computed via the $\ell_2$-norm,
\begin{equation*}
	\Psi_s(k) = ||\b{\psi}_s(k)||_2.
\end{equation*}
The electric torque $T$ is defined as
\begin{equation*}
	T(k) = \frac{1}{\mathrm{pf}}\frac{X_m}{D}\b{\psi}_r\times\b{\psi}_s = \Tfac\b{\psi}_r\times\b{\psi}_s,
\end{equation*}
where $\mathrm{pf},\ X_m,$ and $D$ resemble machine parameters and are summarized under $\Tfac = \frac{1}{pf}\frac{X_m}{D}$. The operator $\times$ is the cross-product and can be replaced by the matrix $\Xi$:
\begin{equation*}
	\b{\psi}_r\times\b{\psi}_s = \b{\psi}_r^\top\Xi\b{\psi}_s,\ \text{where } \Xi = \left[\begin{array}{cc}
		0 & 1 \\
		-1 & 0
	\end{array}\right].
\end{equation*}




\section{SDP relaxations}

\subsection{Problem definition}

\subsection{Solvers}





\section{Conventional FCS-MPC for torque tracking}

\subsection{Problem formulation}

The 1-step cost function from \cite{gey_book} is specified as:
\begin{equation*}
	J(k) = \lambda_T ||T^*(k+1)-T(k+1)||_2 + (1-\lambda_T) || (\Psi_s^*(k+1))^2-(\Psi_s(k+1))^2||_2 + \lambda_u ||\Delta \b{u}(k)||_1,
\end{equation*}
where $T^*(k)$ and $\Psi_s^*(k)$ refer to the Torque- and absolute stator flux references, respectively.
Using an extended horizon of $N \in \mathbb{N}$ leads to the cost function
\begin{equation}\label{eq:2}
	J_N(k) = \sum_{l = k}^{k+N-1} \lambda_T ||T^*(l+1)-T(l+1)||_2^2 + (1-\lambda_T) ||(\Psi_s^*(l+1))^2-(\Psi_s(l+1))^2||_2^2 + \lambda_u ||\Delta \b{u}(l)||_1.
\end{equation}

\subsection{Solving with branch-and-bound algorithms}


