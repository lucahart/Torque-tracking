\documentclass{scrartcl}

% Language setting
\usepackage[english]{babel}

% Set page size and margins
% Replace `letterpaper' with`a4paper' for UK/EU standard size
\usepackage[letterpaper,top=2cm,bottom=2cm,left=3cm,right=3cm,marginparwidth=1.75cm]{geometry}

% Useful packages
\usepackage{amsmath}
\usepackage{amssymb}
\usepackage{graphicx}
\usepackage[colorlinks=true, allcolors=black]{hyperref}
\usepackage{enumitem}
\usepackage{wrapfig}
\usepackage{gensymb}
\usepackage{bbm}
\usepackage{mathtools}
\usepackage{subcaption}
\usepackage{algorithm}
\usepackage{algpseudocode}
\usepackage{tikz,pgfplots}
\usepackage{graphicx}
\usepackage{siunitx}
\usetikzlibrary{positioning}
\usetikzlibrary{calc}
\usepackage{booktabs}
\usepackage{xcolor,calc}
\usetikzlibrary{arrows}
\usetikzlibrary{shapes.geometric}
\DeclarePairedDelimiter\abs{\lvert}{\rvert}%
% \usepackage[framed,numbered,autolinebreaks,useliterate]{mcode}

\newcommand{\dc}{\mathrm{dc}}
\newcommand{\abc}{\mathrm{abc}}
\newcommand{\A}{\mathrm{a}}
\newcommand{\B}{\mathrm{b}}
\newcommand{\C}{\mathrm{c}}
\newcommand{\g}{\mathrm{g}}
\newcommand{\ph}{\mathrm{ph}}
\newcommand{\sw}{\mathrm{sw}}
\newcommand{\s}{\mathrm{s}}
\newcommand{\TDD}{\mathrm{TDD}}
\newcommand{\err}{\mathrm{err}}
\newcommand{\p}{\mathrm{p}}
\newcommand{\U}{\mathrm{u}}
\newcommand{\temp}{\mathrm{temp}}
\newcommand{\PI}{\mathrm{PI}}
\newcommand{\unc}{\mathrm{unc}}
\newcommand{\ctrl}{\mathrm{ctrl}}
\newcommand{\kI}{k_\mathrm{I}}
\newcommand{\kP}{k_\mathrm{P}}
\newcommand{\mavg}{\mathrm{mavg}}
\newcommand{\Sim}{\mathrm{sim}}
\renewcommand{\b}[1]{\boldsymbol{#1}}
\newlength{\fheight}
\newlength{\fwidth}
\newcommand{\I}{\b{I}}
\newcommand{\0}{\b{0}}
\newcommand{\bab}{\mathrm{bab}}
\newcommand{\ed}{\mathrm{ed}}
\newcommand{\ini}{\mathrm{ini}}
\newcommand{\augp}{\mathrm{aug}}
\newcommand{\augs}{\mathrm{aug}}
\newcommand{\opt}{\mathrm{opt}}
\newcommand{\Tfac}{T_\mathrm{factor}}
\renewcommand{\sp}{\text{ }}
\DeclareSIUnit{\pu}{pu}

\DeclareMathOperator*{\argmin}{\mathrm{argmin}}
\definecolor{blueCol}{rgb}{0, 0.447 0.741}

\begin{document}

%\title{}
%\subtitle{}
%\author{}
%
%%\maketitle
%\tableofcontents
%\clearpage

\section{Modelling}

\subsection{Flux Dynamics}
We utilize the discrete time dynamics of the stator and rotor fluxes $\b{\psi}_s$ and $\b{\psi}_r$ from \cite{gey_book}:
\begin{align*}
	\b{\psi}_s(k+1) &= \b{A}_1\b{\psi}_s(k) + \b{B}_1\b{\psi}_r(k) + \b{B}_2\b{u}(k), \\
	\b{\psi}_r(k+1) &= \b{B}_3\b{\psi}_s(k) + \b{A}_2\b{\psi}_r(k),
\end{align*}
where $\b{u}(k) \in \{-1,0,1\}^3$ is the normalized input.
These dynamics can be summarized to
\begin{equation*}
	\b{x}(k+1) = \b{Ax}(k) = \b{Bu}(k),
\end{equation*}
where
\begin{equation*}
	A = \left[\begin{array}{cc}
		\b{A}_1 & \b{B}_1 \\
		\b{B}_3 & \b{A}_2
	\end{array}\right],\ \b{B} = \left[\begin{array}{c}
		\b{B}_2 \\
		\b{0}
	\end{array}\right].
\end{equation*}
The values of the rotor and stator flux at time step $k+l+1$, given the state $x(k)$, all inputs $\b{u}(k),\ \b{u}(k+1),\ldots,\ \b{u}(k+l)$, and $l \geq 2$ can now be computed as
\begin{align}\label{eq:0}
	\b{\psi}_s(k+l+1) &= \left[\b{A}_1\ \b{B}_1\right]\b{A}^{(l-1)}\b{x}(k) + \left[\b{A}_1\ \b{B}_1\right]\b{A}^{(l-2)}\b{Bu}(k) + \\
	&\ + \left[\b{A}_1\ \b{B}_1\right]\b{A}^{(l-3)}\b{Bu}(k+1) + \ldots + \left[\b{A}_1\ \b{B}_1\right]\b{Bu}(k+l-1) + \b{B}_2\b{u}(k+l) \nonumber
\end{align}
and
\begin{align} \label{eq:1}
	\b{\psi}_r(k+l+1) &= \left[\b{B}_3\ \b{A}_2\right]\b{A}^{(l-1)}\b{x}(k) + \left[\b{B}_3\ \b{A}_2\right]\b{A}^{(l-2)}\b{Bu}(k) + \\
	&\ + \left[\b{B}_3\ \b{A}_2\right]\b{A}^{(l-3)}\b{Bu}(k+1) + \ldots + \left[\b{B}_3\ \b{A}_2\right]\b{Bu}(k+l-1). \nonumber
\end{align}

\subsection{Torque and Absolute Stator Flux}

The absolute stator flux is computed via the $\ell_2$-norm,
\begin{equation*}
	\Psi_s(k) = ||\b{\psi}_s(k)||_2.
\end{equation*}
The electric torque $T$ is defined as
\begin{equation*}
	T(k) = \frac{1}{\mathrm{pf}}\frac{X_m}{D}\b{\psi}_r\times\b{\psi}_s = \Tfac\b{\psi}_r\times\b{\psi}_s,
\end{equation*}
where $\mathrm{pf},\ X_m,$ and $D$ resemble machine parameters and are summarized under $\Tfac = \frac{1}{pf}\frac{X_m}{D}$. The operator $\times$ is the cross-product and can be replaced by the matrix $\Xi$:
\begin{equation*}
	\b{\psi}_r\times\b{\psi}_s = \b{\psi}_r^\top\Xi\b{\psi}_s,\ \text{where } \Xi = \left[\begin{array}{cc}
		0 & 1 \\
		-1 & 0
	\end{array}\right].
\end{equation*}



\section{Cost Formulation}

\subsection{Cost Function}

The 1-step cost function from \cite{gey_book} is specified as:
\begin{equation*}
	J(k) = \lambda_T(T^*(k+1)-T(k+1))^2 + (1-\lambda_T)(\Psi_s^*(k+1)-\Psi_s(k+1))^2 + \lambda_u ||\Delta \b{u}(k)||_1,
\end{equation*}
where $T^*(k)$ and $\Psi_s^*(k)$ refer to the Torque- and absolute stator flux references, respectively.
Using an extended horizon of $N \in \mathbb{N}$ leads to the cost function
\begin{equation*}
	J_N(k) = \sum_{l = k}^{k+N-1} \lambda_T(T^*(l+1)-T(l+1))^2 + (1-\lambda_T)(\Psi_s^*(l+1)-\Psi_s(l+1))^2 + \lambda_u ||\Delta \b{u}(l)||_1.
\end{equation*}


\subsection{Matrices for Full Horizon Dynamics}

Next, we will obtain a direct representation of the cost with matrices. Therefore, we introduce the full-horizon variables in Table \ref{tab:0}.
\begin{table}[h!]
	\centering
	\begin{tabular}{ll}
		\toprule
		variable name & variable description \\
		\midrule[\heavyrulewidth]
		$\b{\psi}_s(k) = [\psi_s^\top(k+1),\ \psi_s^\top(k+2),\ldots,\ \psi_s^\top(k+N)]^\top$ & Stator Flux \\
		$\b{\psi}_r(k) = [\psi_r^\top(k+1),\ \psi_r^\top(k+2),\ldots,\ \psi_r^\top(k+N)]^\top$ & Rotor Flux \\
		$\b{T}(k) = [T(k+1),\ T(k+2),\ldots,\ T(k+N)]^\top$ & Torque \\
		$\b{\Psi}_s(k) = [\Psi_s(k+1),\ \Psi_s(k+2),\ldots,\ \Psi_s(k+N)]^\top$ & Absolute stator flux \\
		$\b{T}^*(k) = [T^*(k+1),\ T^*(k+2),\ldots,\ T^*(k+N)]^\top$ & Torque reference \\
		$\b{\Psi}^*_s(k) = [\Psi_s^*(k+1),\ \Psi^*_s(k+2),\ldots,\ \Psi_s^*(k+N)]^\top$ & Absolute stator flux reference \\
		$\b{U}(k) = [\b{u}^\top(k),\ \b{u}^\top(k+1),\ldots,\ \b{u}^\top(k+N-1)]^\top$ & Inputs \\
		\bottomrule
	\end{tabular}
	\caption{Full horizon variables}
	\label{tab:0}
\end{table}
While the references $\b{T}^*(k),\ \b{\Psi}^*(k)$ are given, we will have to find a formulation to compute the actual values of $\b{\Psi}_s(k)$ and $\b{T}(k)$.\\
Utilizing \eqref{eq:0} and \eqref{eq:1} we get the matrix representations
\begin{equation*}
		\psi_s(k+l+1) = \left[	\b{A}_1 \b{B}_1\right]\b{A}^{(l-1)}\b{x}(k) + \left[\begin{array}{ccccc}
		\left[\b{A}_1\ \b{B}_1\right]\b{A}^{(l-2)}\b{B} & \left[\b{A}_1\ \b{B}_1\right]\b{A}^{(l-3)}\b{B} & \ldots & \left[\b{A}_1\ \b{B}_1\right]\b{B} & \b{B}_2
		\end{array} \right]\b{U}(k)
\end{equation*}
and
\begin{equation*}
	\psi_r(k+l+1) = \left[	\b{B}_3 \b{A}_2\right]\b{A}^{(l-1)}\b{x}(k) + \left[\begin{array}{ccccc}
		\left[	\b{B}_3 \b{A}_2\right]\b{A}^{(l-2)}\b{B} & \left[	\b{B}_3 \b{A}_2\right]\b{A}^{(l-3)}\b{B} & \ldots & \left[	\b{B}_3 \b{A}_2\right]\b{B} & \b{0}
	\end{array} \right]\b{U}(k).
\end{equation*}
This can further be used to obtain the full horizon of the stator flux
\begin{align*}
	\b{\psi}_s(k) &=  \left[	\b{A}_1 \b{B}_1\right] \left[\begin{array}{c}
		\I \\
		\b{A} \\
		\b{A}^2 \\
		\vdots \\
		\b{A}^{N-1}
	\end{array}\right] \b{x}(k) + 
\left[\begin{array}{ccccc}
	\b{B}_2 & & & & \b{0} \\
	\left[\b{A}_1\ \b{B}_1\right]\b{B} & \b{B}_2 & & & \\
	\left[\b{A}_1\ \b{B}_1\right]\b{AB} & \left[\b{A}_1\ \b{B}_1\right]\b{B} & \b{B}_2 && \\
	\vdots & \vdots & \vdots & \ddots & \\
	\left[\b{A}_1\ \b{B}_1\right]\b{A}^{N-2}\b{B} & \left[\b{A}_1\ \b{B}_1\right]\b{A}^{N-3}\b{B} & \left[\b{A}_1\ \b{B}_1\right]\b{A}^{N-4}\b{B} & \cdots & \b{B}_2 \\
\end{array}\right]\b{U}(k) \\
	&=: \b{\Gamma}_s\b{x}(k) + \b{\Upsilon}_s\b{U}(k)
\end{align*}
and the rotor flux
\begin{align*}
	\b{\psi}_r(k) &=  \left[\b{B}_3 \b{A}_2\right] \left[\begin{array}{c}
		\I \\
		\b{A} \\
		\b{A}^2 \\
		\vdots \\
		\b{A}^{N-1}
	\end{array}\right] \b{x}(k) + 
	\left[\begin{array}{ccccc}
		\b{0} & & & & \b{0} \\
		\left[\b{B}_3 \b{A}_2\right] \b{B} & \b{0} & & & \\
		\left[\b{B}_3 \b{A}_2\right] \b{AB} & \left[\b{B}_3 \b{A}_2\right] \b{B} & \b{0} && \\
		\vdots & \vdots & \vdots & \ddots & \\
		\left[\b{B}_3 \b{A}_2\right] \b{A}^{N-2}\b{B} & \left[\b{B}_3 \b{A}_2\right] \b{A}^{N-3}\b{B} & \left[\b{B}_3 \b{A}_2\right] \b{A}^{N-4}\b{B} & \cdots & \b{0} \\
	\end{array}\right]\b{U}(k)\\
	&=: \b{\Gamma}_r\b{x}(k) + \b{\Upsilon}_r\b{U}(k).
\end{align*}



\subsection{Matrices for Full Horizon Torque and Stator Flux}
%Stator flux:
%\begin{align*}
%	\Psi_s^2(k) = ||\b{\psi}_s(k)||_2^2 = \b{\psi}_s^\top(k)\b{\psi}_s(k)
%\end{align*}
%Full horizon stator flux:
%\begin{align*}
%	\b{\Psi}_s^2(k) = \left[\begin{array}{c}
%		\Psi_s^2(k)\\
%		\Psi_s^2(k+1)
%	\end{array}\right],
%\end{align*}
%where the squared-operation is an element-wise operation.

Torque:
\begin{align*}
	\b{T}(k) &= \left[\begin{array}{c}
		T(k+1) \\
		T(k+2) \\
		\vdots \\
		T(k+N+1)
	\end{array}\right]
	= \Tfac\left[\begin{array}{c}
		\b{\psi}_r(k+1)\Xi\b{\psi}_s(k+1) \\
		\b{\psi}_r(k+2)\Xi\b{\psi}_s(k+2) \\
		\vdots \\
		\b{\psi}_r(k+N+1)\Xi\b{\psi}_s(k+N+1)
	\end{array}\right] \\
	&= \Tfac\b{\psi}_r(k) \left[\begin{array}{ccc}
		\Xi & & \\
		& \ddots & \\
		& & \Xi
	\end{array}\right]\b{\psi}_s(k) := \Tfac\b{\psi}_r(k) \Xi \b{\psi}_s(k)
\end{align*}


\subsection{Final cost formulation}




\bibliographystyle{ieeetr}
\bibliography{bibliography}



\end{document}